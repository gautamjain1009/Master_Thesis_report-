%!TEX root = ../report.tex

\begin{document}
    \begin{abstract}
        The task of lane detection has significant importance in the field of autonomous driving. An efficient lane detection algorithm can contribute toward optimal planning for an autonomous vehicle. A conventional modular autonomous driving stack consists of modules like perception, control, and planning which are interrelated to each other. All the information perceived from the sensors mounted on an autonomous vehicle is processed by these modules to predict a decision for an autonomous vehicle.  Whereas in end-to-end autonomous driving all of these functional modules are replaced by a large neural network that acts like a black box to predict the decisions for an autonomous vehicle. A modern end-to-end driving policy is composed of a large neural network that predicts multiple tasks like object detection, depth estimation, 3D lane detection, object tracking, and traffic sign detection along with predicting the driving commands. The prediction of driving commands is considered as the main task, whereas the other predicted tasks are categorized under auxiliary tasks. Utilizing the supervision signals from the auxiliary tasks we can make a more efficient end-to-end driving policy. Generally, HD maps are utilized in modular autonomous driving stacks along with other sensory information to plan and act. Maintenance and updating these HD maps is a costly and labor-intensive process. The current research in end-to-end autonomous driving is moving towards a Lidar-free way of perceiving the surroundings, in which a local HD map of the surrounding is created using only the surround-view camera setup. This work can be seen as an initial step toward building a local HD map where a monocular camera image is used to predict robust 3D lane line points. Inspired from GenLaneNet\cite{guo2020gen} and semi-local 3D LaneNet \cite{DBLP:journals/corr/abs-2011-01535} we have proposed a dual stage anchor-less semi-local 3D lane detector. We have utilized the state-of-the-art segmentation-based 2D lane detection architectures to obtain robust binary lane segmentation masks for the first stage of the pipeline. By replacing the ERFNet\cite{Romera2018ERFNetER} based binary lane segmentation architecture utilized in GenLaneNet\cite{guo2020gen} with more complex architectures, we have shown that an efficient binary lane segmentation solution can enhance the performance of a dual-stage 3D lane detector. TuSimple \cite{Tusimple}, CuLane \cite{pan2018SCNN} and Apollo Synthetic 3D lane dataset \cite{guo2020gen} is used to train the first stage of the pipeline, whereas the semi-local anchor-less 3D lane detector is trained on Apollo Synthetic 3D lane dataset \cite{guo2020gen}. This work can be extended to an end-to-end autonomous driving policy in which 3D lane detection is utilized as an auxiliary task. 
    \end{abstract}
\end{document}
